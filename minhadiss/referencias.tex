%%%%%%%%%%%%%%%%%%%%%%%%%%%
\begin{thebibliography}{99}
%\advance\leftskip by 1.5 %em
%\advance\itemindent by -1%.5 em
%\addtocontents{toc}{\protect\vspace{1ex}}
%\addcontentsline{toc}{chapter}{\bf\hspace{0.47cm} Referências}
%\addtocontents{toc}{\protect\vspace{1ex}}


% Referências

\bibitem{goswani03} GOSWANI, D. Optical pulse shaping approaches to coherent control. {\it PHYSICS REPORTS}, v.374, p.385-481 (2003).

\bibitem{warren93} WARREN, W.; RABITZ, H.; DAHLEH, M. Coherent control of quantum dynamics: the dream is alive. {\it SCIENCE}, v.259, p.1581-1589 (1993).

\bibitem{rabitz00} RABITZ, H. {\it et al.} Whither the future of controlling quantum phenomena. {\it SCIENCE}, v.288, p.824-828 (2000).

\bibitem{rabitz_zhu00}  RABITZ, H.; ZHU, W. Optimal control of molecular motion: design, implementation, and inversion. {\it ACCOUNTS OF CHEMICAL RESEARCH}, v.33, p.572-578 (2000).

\bibitem{kawashima95} KAWASHIMA, H.; WEFERS, M. M.; NELSON, K. A. Femtosecond pulse shaping, multiple-pulse spectroscopy, and optical control. {\it ANNUAL REVIEW OF PHYSICAL CHEMISTRY}, v.46, p.627-656 (1995).

\bibitem{assion98} ASSION, A. {\it et al.} Control of chemical reactions by feedback-optimized phase-shaped femtosecond laser pulses. {\it SCIENCE}, v.282, p.919-922 (1998).

\bibitem{silberberg09} SILBERBERG Y. Quantum Coherent Control for Nonlinear Spectroscopy and Microscopy. {\it ANNUAL REVIEW OF PHYSICAL CHEMISTRY}, v.60, p.277-292 (2009).

\bibitem{vivie-riedle01} DE VIVIE-RIEDLE, R.; KURTZ, L.; HOFMANN, A. Coherent control for ultrafast photochemical reactions. {\it PURE AND APPLIED CHEMISTRY}, v.73, p.525-528 (2001).

\bibitem{gordon97} GORDON, R. J.; RICE, S. A. Active control of the dynamics of atoms and molecules. {\it ANNUAL REVIEW OF PHYSICAL CHEMISTRY}, v.48, p.601-641, (1997).

\bibitem{tannor85} TANNOR D. J.; RICE S. A. Control of selectivity of chemical-reaction via control of wave packet evolution. {\it JOURNAL OF CHEMICAL PHYSICS}, v.83, p.5013-5018 (1985).

\bibitem{baumert94} BAUMERT, T.; GERBER, G. Fundamental Interactions of Molecules (Na$_2$, Na$_3$) with Intense Femtosecond Laser Pulses. {\it ISRAEL JOURNAL OF CHEMISTRY}, v.34, p.103-114 (1994).

\bibitem{shapiro86} SHAPIRO, M.; BRUMER, P. Laser control of product quantum state populations in unimolecular reactions. {\it JOURNAL OF CHEMICAL PHYSICS}, v.84, p.4103-4104 (1986).

\bibitem{zhu95} ZHU, L. C. {\it et al.} Coherent laser control of the product distribution obtained in the photoexcitation of HI. {\it SCIENCE}, v.270, p.77-80 (1995).

\bibitem{zhu97} ZHU, L. C. {\it et al.} Effect of resonances on the coherent control of the photoionization and photodissociation of HI and DI. {\it PHYSICAL REVIEW LETTERS}, v.79, p.4108-4111 (1997).

\bibitem{larsen99} LARSEN, J. J.; WENDT-LARSEN I.; STAPELFELDT H. Controlling the branching ratio of photodissociation using aligned molecules. {\it PHYSICAL REVIEW LETTERS}, v.83, p.1123-1126 (1999).

\bibitem{brixner01} BRIXNER, T.; GERBER, G. Femtosecond polarization pulse shaping. {\it OPTICS LETTERS}, v.26, p.557-559 (2001).

\bibitem{brixner02} BRIXNER, T. {\it et al.} Generation and characterization of polarization-shaped femtosecond laser pulses. {\it APPLIED PHYSICS B - LASERS AND OPTICS}, v.74, p.133-144 (2002).

\bibitem{mayer64} MAYER, G.; GIRES, F. Action d'une onde lumineuse intense sur l'indice de réfraction des liquides. {\it COMPTES RENDUS HEBDOMADAIRES DES SEANCES DE L ACADEMIE DES SCIENCES}, v.258, p.2039-2042 (1964).

\bibitem{duguay69} DUGUAY, M. A.; HANSEN, J. W. An ultrafast light gate. {\it APPLIED PHYSICS LETTERS}, v.15, p.192-194 (1969).

\bibitem{zhong08} ZHONG, Q.; FOURKAS, J. T. Optical Kerr Effect Spectroscopy of Simple Liquids. {\it JOURNAL OF PHYSICAL CHEMISTRY B}, v.112, p.15529-15539 (2008).

\bibitem{heisler06tese} HEISLER, I. A. Espectroscopia resolvida no tempo: caracterização de pulsos curtos, dinâmica molecular em líquidos, modelamento de luz incoerente. {\it TESE (DOUTORADO) - UFRGS} (2006).

\bibitem{zhux05} ZHU, X.; FARRER, R. A.; FOURKAS, J. T. Optical Kerr effect spectroscopy using time-delayed pairs of pump pulses with orthogonal polarizations. {\it JOURNAL OF PHYSICAL CHEMISTRY B}, v.109, p.8481-8488 (2005).

\bibitem{heisler05} HEISLER I. A.; CORREIA R. R. B.; BUCKUP T. Time-resolved optical Kerr-effect investigation on CS2/polystyrene mixtures. {\it JOURNAL OF CHEMICAL PHYSICS},  v.123, p.054509 (2005).

\bibitem{heisler06} HEISLER, I. A.; CORREIA R. R. B.; CUNHA S. L. S. Molecular dynamics investigation with the time resolved optical Kerr effect on the CS2-C6H6 mixtures. {\it JOURNAL OF CHEMICAL PHYSICS}, v.125, p.184503 (2006).

\bibitem{brixner04} BRIXNER, T. {\it et al.} Quantum control by ultrafast polarization shaping. {\it PHYSICAL REVIEW LETTERS}, v.92, p.208301 (2004).

\bibitem{polachek06} POLACHEK, L.; ORON, D.; SILBERBERG, Y. Full control of the spectral polarization of ultrashort pulses. {\it OPTICS LETTERS}, v.31, p.631-633 (2006).

\bibitem{sheik-bahae89} SHEIK-BAHAE, M.; SAID, A. A.; VAN STRYLAND, E. W. High-sensitivity, single-beam n2 measurements. {\it OPTICS LETTERS}, v.14, p.955-957 (1989).

\bibitem{sheik-bahae90} SHEIK-BAHAE, M. {\it et al.} Sensitive measurement of optical nonlinearities using a single beam. {\it IEEE JOURNAL OF QUANTUM ELECTRONICS}, v.26, p.760-769 (1990).

\bibitem{xia94} XIA, T. {\it et al.} Eclipsing Z-scan measurement of $\lambda/10^4$ wave-front distortion.
 {\it OPTICS LETTERS}, v.19, p.317-319 (1994).

\bibitem{tuna} BRAUN, L. F. M. Medida e interpretação de sinais optogalvânicos resolvidos no tempo. {\it TESE (DOUTORADO) - UFRGS} (2008).

%\bibitem{penning28} PENNING, F. M. Demonstratie van een nieuw photoelectrisch effect. {\it Physica}, v.8, p.137-140 (1928).

\bibitem{sasso90} SASSO, A. {\it et al.} High-resolution and doppler-limited laser investigation of atomic
oxygen production in o-2-noble gases radiofrequency discharges. {\it JOURNAL OF CHEMICAL PHYSICS}, v.93, p.7774-7779 (1990).

\bibitem{chung90} CHUNG, Y. C. Frequency-locked 1.3$\mu$m and 1.5$\mu$m semiconductor-lasers
for lightwave systems applications. . {\it JOURNAL OF LIGHTWAVE TECHNOLOGY}, v.8, p.869-876 (1990).

\bibitem{may85} MAY, R. D. Spatial characteristics of the optogalvanic effect in a striated rare-gas
discharges. {\it JOURNAL OF APPLIED PHYSICS}, v.58, p. 1169-1176 (1985).

\bibitem{ben-amar84} BEN-AMAR, A. {\it et al.} Observation of Penning ionization in sr/ne discharge by the optogalvanic effect. {\it APPLIED OPTICS}, v.23, p.4529-4531 (1984).

\bibitem{katsonis80} KATSONIS, K.; DRAWIN, H. W. Transition-probabilities for argon(I). {\it JOURNAL OF QUANTITATIVE SPECTROSCOPY \& RADIATIVE TRANSFER}, v.23, p.1-55 (1980).

\bibitem{whaling02} Whaling, W. {\it et al.} Argon I lines produced in a hollow cathode source, 332nm to 5865nm. {\it JOURNAL OF RESEARCH OF THE NATIONAL INSTITUTE OF STANDARDS AND TECHNOLOGY}, v.107, p.149-169 (2002).

\bibitem{tournois97} TOURNOIS, P. Acousto-optic programmable dispersive filter for adaptive compensation of group delay time dispersion in laser systems. {\it OPTICS COMMUNICATIONS}, v.140, p.245-249 (1997).

\bibitem{tournois00} TOURNOIS, P. {\it et al.} Amplitude and phase control of ultrashort pulses by use of an acousto-optic programmable dispersive filter: pulse compression and shaping. {\it OPTICS LETTERS}, v.25, p.575-577 (2000).

\bibitem{dixon67} DIXON, R. W. Acoustic Diffraction of Light in Anisotropic Media. {\it IEEE JOURNAL OF QUANTUM ELECTRONICS}, v.QE 3, p.85-93 (1967).

\bibitem{harris69}  HARRIS, S. E.; WALLACE, R. W. Acousto-optic tunable filter. {\it JOURNAL OF THE OPTICAL SOCIETY OF AMERICA}, v.59, p.744-747 (1969).

\bibitem{bei04} BEI, L.  Acousto-optic tunable filters: fundamentals and applications as applied to chemical analysis techniques. {\it PROGRESS IN QUANTUM ELECTRONICS}, v.28, p.67-87 (2004).

\bibitem{vebber07} VEBBER, G. C. Desenvolvimento e caracterização de um modelador óptico programável. {\it DISSERTAÇÃO (MESTRADO) - UFRGS} (2007).

\bibitem{denk90} DENK, W.; STRICKLER, J. H.; WEBB, W. W. Two-photon laser scanning fluorescence microscopy. {\it SCIENCE}, v.248, p.73-76 (1990).

\bibitem{hirose88} HIROSE, C.; MASAKI, T. A theory of electric field and charge-density distributions inside a cylindrical hollow-cathode. {\it APPLIED SPECTROSCOPY}, v.42, p.811–815 (1988).

\bibitem{zeidler01} ZEIDLER, D. {\it et al.} Evolutionary algorithms and their application to optimal control studies. {\it PHYSICAL REVIEW A}, v.64, p.023420 (2001).

\bibitem{judson92} JUDSON, R. S.; RABITZ, H. Teaching lasers to control molecules. {\it PHYSICAL REVIEW LETTERS}, v.68, p.1500-1503 (1992).

\end{thebibliography}





