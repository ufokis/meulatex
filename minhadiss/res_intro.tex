\paragraph{}

O controle coerente da dinâmica quântica é um ideal perseguido por cientistas mesmo antes da invenção do laser, uma ferramenta básica para sua concretização. Ele tem inspirado tentativas de controle, por exemplo, sobre: a formação e a dissociação seletivas em reações químicas; o movimento de pacotes de onda atômicos e moleculares em processos fotoquímicos; e, mais recentemente, a detecção ou excitação seletiva em técnicas espectroscópicas. O trabalho que  apresentamos aqui envolve os dois últimos exemplos de controle.

Após revisarmos, de maneira geral e direcionada, o campo abrangente de controle quântico no Capítulo \ref{cap1}, relatamos no Capítulo \ref{cap2} um experimento de controle sobre um processo fotoquímico: o efeito optogalvânico sobre uma descarga de catodo oco. No Capítulo \ref{cap3}, estabelecemos algumas bases conceituais para experimentos de controle em duas técnicas de óptica não linear, espectroscopia OKE e Z-scan, e o uso de modelagem óptica de polarização. As perspectivas projetam desenvolvimentos experimentais relativos aos experimentos do Capítulo \ref{cap2} e, principalmente, do Capítulo \ref{cap3}. Por último, no Capítulo 








*
**



Por último, nNPor último, n2 e controle soassuntos do sudalém de  sedimenta
