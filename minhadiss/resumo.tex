%\begin{abstract}
%\paragraph*{}
\chapter*{Resumo}
\paragraph{}

Realizamos neste trabalho uma revisão na área de controle coerente, com enfoque em con-
trole otimizado experimental e aplicações em técnicas de espectroscopia não linear. Além
disso, descrevemos um experimento de controle sobre o sinal optogalvânico de uma descarga
de catodo oco, considerando desde aspectos fundamentais da descarga até a análise de re-
sultados obtidos. Por fim, apresentamos as técnicas de espectroscopia OKE e de Z-scan,
assim como uma nova classe de modeladores ópticos de pulsos (de polarização), a fim de
fornecer um panorama básico para a principal perspectiva desse trabalho de doutorado:
experimentos de controle coerente em técnicas de óptica não linear aplicando modelagem
de polarização.

%\end{abstract}
