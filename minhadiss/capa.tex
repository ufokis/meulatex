\begin{titlepage}
\begin{center}\Large
{UNIVERSIDADE FEDERAL DO RIO GRANDE DO SUL \\
INSTITUTO DE FÍSICA}
\end{center}
\vfill
\begin{center}\Large
\renewcommand{\thefootnote}{\fnsymbol{footnote}}
\setcounter{footnote}{1}
{\bf Controle Coerente em Espectroscopia Optogalvânica e Óptica Não Linear}%
$\footnote[1]{Trabalho financiado pelo Conselho Nacional de
Desenvolvimento Científico e Tecnológico (CNPq).}$
\end{center}

\bigskip

\begin{center}
\Large
Guilherme Cañete Vebber
\end{center}
\bigskip

\vfill
\hfill
\begin{minipage}[b]{0.4\textwidth}
Exame de Qualificação ao Doutorado realizado sob a orientação do Professor Dr. Ricardo Rego Bordalo Correia e apresentado ao Instituto de Física da UFRGS, em preenchimento parcial dos requisitos para a obtenção do título de Doutor em Física.
\end{minipage}
\setcounter{footnote}{0}
\renewcommand{\thefootnote}{\arabic{footnote}}
\vfill
\begin{center}
{Porto Alegre, dezembro de 2009.}
\end{center}
\end{titlepage}
